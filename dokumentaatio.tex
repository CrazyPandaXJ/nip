% NIP - project documentation
% Mikko Korpela, Antti Rasinen, Janne Toivola 2004
% $Id: dokumentaatio.tex,v 1.2 2004-08-20 09:52:17 jatoivol Exp $

\documentclass[12pt,a4paper]{report}
%\bibliographystyle{unsrt}
\setlength{\topmargin}{0mm}
\setlength{\oddsidemargin}{0mm}
\setlength{\textheight}{22cm}
\setlength{\textwidth}{16cm}
\usepackage[T1]{fontenc}
\usepackage[latin1]{inputenc}
\usepackage[english]{babel}
\usepackage[dvips]{graphicx}
\usepackage{amsmath}
\usepackage{subfigure}
\usepackage{tocbibind}
\usepackage{fancyvrb}
%\newcommand{\argmax}{\mathrm{argmax}}
%\newcommand{\argmin}{\mathrm{argmin}}
%\newcommand{\define}{\stackrel{\mathrm{def}}{=}}

%--------------------
% Place this in the preamble of your LaTeX file:
% (c) Jaakko Hollmen, 2002

\usepackage{ifthen}

% Declare the variable doublespaced
\newboolean{doublespaced}

% Comment one of the following lines to either select
% doublespaced or singlespaced:
\setboolean{doublespaced}{false}
%\setboolean{doublespaced}{true}

\ifthenelse{\boolean{doublespaced}}
{
  % Double spaced text if variable "doublespaced" is true:
  \renewcommand{\baselinestretch}{1.5}
  \normalsize % necessary to execute the previous thing
}
{
  % This is to be executed if doublespace is false:
  % else = do nothing
}
%--------------------

\title{NIP}
\author{Antti Rasinen}
\author{Mikko Korpela 54919L}
\author{Janne Toivola 55173U}
\begin{document}

\pagestyle{empty}
\setlength{\parindent}{0mm}
\setlength{\parskip}{3mm}

\large
\textbf{The NIP project}\\

\vspace{45mm}

\begin{centering}
\huge
\textbf{System description}\\ % Any better names for this?
\end{centering}

\parbox{5cm}{\ }
\parbox{1em}{\vskip8cm}

\normalsize
\vspace{5mm}
\begin{tabbing}
The Team:\= Antti Rasinen\\
         \> Mikko Korpela\\
         \> Janne Toivola\\
\vspace{5mm}

\end{tabbing}
%----------------------
Date: \today
%----------------------
\eject\newpage

\pagestyle{plain}

\tableofcontents

%\newpage
%
%\listoffigures

\newpage
\chapter{General description}

Some introduction here...

\section{Overview}

Some outline of the system...

\newpage
\chapter{Subsystem descriptions}
\section{Variables}
\subsection{General}

The idea and mathematical formulas...

\subsection{The functionality description}

\subsubsection{The data structure}
The data type ''Variable'' defined in \verb+Variable.h+ is a pointer 
to a struct which has the following fields: 
\begin{description}
\item[symbol] is a short string used to identify the variable. Maximum
length is defined as \textbf{VAR\_SYMBOL\_LENGTH} in
\verb+Variable.h+. To ensure the correct operation of the system, the 
symbol should be unique.

\item[name] is a little longer name to describe the variable: it is
called {\it label} in the Hugin net files. This is NOT used for 
identification of the variables and it doesn't need to be unique. 
The maximum length is defined as \textbf{VAR\_NAME\_LENGTH} in 
\verb+Variable.h+.

\item[statenames] is an array of strings containing names for all
possible states of the variable. The identification of data relies on
these strings. The size of the statenames array i.e. number of
strings should be the same as the number of states indicated by
cardinality field described below.

\item[cardinality] is a positive integer which tells how many states
the variable can have. It plays an important role in all calculations
to it better be correct.

\item[id] is an {\it unsigned long} used for identification inside the
system. It is set automatically by the \verb+new_variable+ function
and should not be tampered after that.

\item[likelihood] is a {\it double} array describing the probability
distribution of the states of the variable. It is used only for
entering evidence into the system and likelihood arrays of the latent
variables remain filled with ones.

\item[previous] is used to describe the repetitive structure in
timeslice models. It is a pointer to the variable which this variable
will replace in previous timeslice. The pointer is {\it NULL} if the
model has no repetitive structure.

\item[next] is a same kind of pointer as \textbf{previous} above, but
this one tells which variable will be replaced in the next timeslice.
The \textbf{previous} and \textbf{next} pointers should be symmetrical
i.e. if \verb+a->next == b+ then \verb+b->previous == a+.
\end{description}

% FIXME: add the description of variable list structure here

\subsubsection{The functions}
The functions provided in \verb+Variable.c+ are
%The interface + how the stuff works...
\begin{description}
\item[]

\item[]

\item[]

\item[]

\item[]

\item[]

\item[]

\item[]

\item[]

\item[]

\item[]

\item[]

\item[]

\item[]

\item[]

\item[]

\item[]

\end{description}


\subsection{Other possibilities}

How the stuff could also work...


\newpage
\chapter{Subsystem descriptions}
\section{Graphs}
\subsection{General}

The idea and mathematical formulas...


\subsection{The functionality description}
%The interface + how the stuff works...
\subsubsection{The data structures}

\subsubsection{The functions}


\subsection{Other possibilities}

How the stuff could also work...


\newpage
\chapter{Subsystem descriptions}
\section{Potentials}
\subsection{General}

The idea and mathematical formulas...

\subsection{The functionality description}
%The interface + how the stuff works...
\subsubsection{The data structure}

\subsubsection{The functions}


\subsection{Other possibilities}

How the stuff could also work...


\newpage
\chapter{Subsystem descriptions}
\section{Join trees} % Cliques and sepsets
\subsection{General}

The idea and mathematical formulas...

\subsection{The functionality description}
%The interface + how the stuff works...
\subsubsection{The data structures}

\subsubsection{The functions}


\subsection{Other possibilities}

How the stuff could also work...


\newpage
\chapter{Subsystem descriptions}
\section{The Hugin net file parser}
\subsection{General}

The idea and mathematical formulas...

\subsection{The functionality description}

The interface + how the stuff works...

\subsection{Other possibilities}

How the stuff could also work...


\newpage
\chapter{Subsystem descriptions}
\section{The timeslice handler} % Some system which handles time series?
\subsection{General}

The idea and mathematical formulas...

\subsection{The functionality description}

The interface + how the stuff works...

\subsection{Other possibilities}

How the stuff could also work...



\newpage
\chapter{System analysis}
\section{The parser}

\section{The graph}

\section{The join tree}

\section{The timeslice-monster}


\newpage
\chapter{Comments}


\newpage
%\appendix
%\newpage
%\bibliography{dokumentaatio}

\end{document}
