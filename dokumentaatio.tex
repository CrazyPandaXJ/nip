% NIP - project documentation
% Mikko Korpela, Antti Rasinen, Janne Toivola 2004
% $Id: dokumentaatio.tex,v 1.1 2004-08-19 12:39:12 jatoivol Exp $

\documentclass[12pt,a4paper]{report}
%\bibliographystyle{unsrt}
\setlength{\topmargin}{0mm}
\setlength{\oddsidemargin}{0mm}
\setlength{\textheight}{22cm}
\setlength{\textwidth}{16cm}
\usepackage[T1]{fontenc}
\usepackage[latin1]{inputenc}
\usepackage[english]{babel}
\usepackage[dvips]{graphicx}
\usepackage{amsmath}
\usepackage{subfigure}
\usepackage{tocbibind}
\usepackage{fancyvrb}
%\newcommand{\argmax}{\mathrm{argmax}}
%\newcommand{\argmin}{\mathrm{argmin}}
%\newcommand{\define}{\stackrel{\mathrm{def}}{=}}

%--------------------
% Place this in the preamble of your LaTeX file:
% (c) Jaakko Hollmen, 2002

\usepackage{ifthen}

% Declare the variable doublespaced
\newboolean{doublespaced}

% Comment one of the following lines to either select
% doublespaced or singlespaced:
\setboolean{doublespaced}{false}
%\setboolean{doublespaced}{true}

\ifthenelse{\boolean{doublespaced}}
{
  % Double spaced text if variable "doublespaced" is true:
  \renewcommand{\baselinestretch}{1.5}
  \normalsize % necessary to execute the previous thing
}
{
  % This is to be executed if doublespace is false:
  % else = do nothing
}
%--------------------

\title{NIP}
\author{Antti Rasinen}
\author{Mikko Korpela 54919L}
\author{Janne Toivola 55173U}
\begin{document}

\pagestyle{empty}
\setlength{\parindent}{0mm}
\setlength{\parskip}{3mm}

\large
\textbf{The NIP project}\\

\vspace{45mm}

\begin{centering}
\huge
\textbf{Users manual}\\
\end{centering}

\parbox{5cm}{\ }
\parbox{1em}{\vskip8cm}

\normalsize
\vspace{5mm}
\begin{tabbing}
The Team:\= Antti Rasinen\\
         \> Mikko Korpela\\
         \> Janne Toivola\\
\vspace{5mm}

\end{tabbing}
%----------------------
Date: \today
%----------------------
\eject\newpage

\pagestyle{plain}

\tableofcontents

%\newpage
%
%\listoffigures

\newpage
\chapter{General description}

Some introduction here...

\section{Overview}

Some outline of the system...

\newpage
\chapter{Subsystem descriptions}
\section{Variables}
\subsection{General}

The idea and mathematical formulas...

\subsection{The functionality description}

The interface + how the stuff works...

\subsection{Other possibilities}

How the stuff could also work...


\newpage
\chapter{Subsystem descriptions}
\section{Graphs}
\subsection{General}

The idea and mathematical formulas...

\subsection{The functionality description}

The interface + how the stuff works...

\subsection{Other possibilities}

How the stuff could also work...


\newpage
\chapter{Subsystem descriptions}
\section{Potentials}
\subsection{General}

The idea and mathematical formulas...

\subsection{The functionality description}

The interface + how the stuff works...

\subsection{Other possibilities}

How the stuff could also work...


\newpage
\chapter{Subsystem descriptions}
\section{Join trees} % Cliques and sepsets
\subsection{General}

The idea and mathematical formulas...

\subsection{The functionality description}

The interface + how the stuff works...

\subsection{Other possibilities}

How the stuff could also work...


\newpage
\chapter{Subsystem descriptions}
\section{The Hugin net file parser}
\subsection{General}

The idea and mathematical formulas...

\subsection{The functionality description}

The interface + how the stuff works...

\subsection{Other possibilities}

How the stuff could also work...


\newpage
\chapter{Subsystem descriptions}
\section{The timeslice handler} % Some system which handles time series?
\subsection{General}

The idea and mathematical formulas...

\subsection{The functionality description}

The interface + how the stuff works...

\subsection{Other possibilities}

How the stuff could also work...



\newpage
\chapter{System analysis}
\section{The parser}

\section{The graph}

\section{The join tree}

\section{The timeslice-monster}


\newpage
\chapter{Comments}


\newpage
%\appendix
%\newpage
%\bibliography{dokumentaatio}

\end{document}
